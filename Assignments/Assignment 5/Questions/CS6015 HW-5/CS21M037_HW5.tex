\documentclass[solution,addpoints,12pt]{exam}
\usepackage{amsmath}
\usepackage{amsthm}
\usepackage{amssymb}
\usepackage{tikz}
\usepackage{animate}
\usepackage{hyperref}

\newtheorem{theorem}{Theorem}
\newtheorem{lemma}[theorem]{Lemma}


\newenvironment{Solution}{\begin{EnvFullwidth}\begin{solution}}{\end{solution}\end{EnvFullwidth}}



\printanswers
%\unframedsolutions
\pagestyle{headandfoot}

%%%%%%%%%%%%%%%%%%%%%%%%%%%%%%%%%%%%%%%%%%%%%%%%%%%%%%
%%%%%%%%%%%%%%%%%%% INSTRUCTIONS %%%%%%%%%%%%%%%%%%%%%
% * Fill in your name and roll number below

% * Answer in place (after each question)

% * Use \begin{solution} and \end{solution} to typeset
%   your answers.
%%%%%%%%%%%%%%%%%%%%%%%%%%%%%%%%%%%%%%%%%%%%%%%%%%%%%%
%%%%%%%%%%%%%%%%%%%%%%%%%%%%%%%%%%%%%%%%%%%%%%%%%%%%%%

% Fill in the details below
\def\studentName{\textbf{Name: TODO}}
\def\studentRoll{\textbf{Roll No: TODO}}

\firstpageheader{CS 6015 (LARP) - Homework 5 (\numpoints~Marks)}{}{\studentName,\studentRoll}
\firstpageheadrule

\newcommand{\brac}[1]{\left[ #1 \right]}
\newcommand{\curly}[1]{\left\{ #1 \right\}}
\newcommand{\paren}[1]{\left( #1 \right)}
\newcommand{\card}[1]{\left\lvert #1 \right\rvert}

\begin{document}

\noindent \textbf{Honor code}: I pledge on my honor that: I have completed all steps in the below homework on my own, I have not used any unauthorized materials while completing this homework, and I have not given anyone else access to my homework.
\\~\\~\\
\begin{flushright}
\textbf{Name and Signature}

\end{flushright}


\begin{questions}

\question[1] Have you read and understood the honor code?
\begin{solution}

\end{solution}

\uplevel{\textbf{Eigenstory: Special Properties}}

\question[1] 
\begin{parts}
\part Give a $3\times3$ matrix such that any two of it's eigenvectors corresponding to distinct eigenvalues are independent. Also, write the eigenvectors and their corresponding eigenvalue. 
\begin{solution}

\end{solution}
\part Give a $3\times3$ matrix (not Identity matrix) such that any two of it's eigenvectors corresponding to non-distinct eigenvalues are independent. Again, write the eigenvectors and their corresponding eigenvalue. 
\begin{solution}

\end{solution}
\end{parts}

\question[2] 
\begin{parts}
\part Let $A$ be a $K \times K$ square matrix. Prove that a scalar $\lambda $ is an eigenvalue of $A$ if and only if it is an eigenvalue of $A^{\top}$.
\begin{solution}

\end{solution}
\part The product of the eigenvalues of a matrix is equal to its determinant. Prove that the diagonal elements of a triangular matrix are equal to its eigenvalues.
\begin{solution}

\end{solution}
\end{parts}

\question[2] 
Let $A$ be a $K \times K$ matrix. Let $\lambda_{k}$ be one of the eigenvalues of $A$. Then prove that the geometric multiplicity of $\lambda_{k}$ is less than or equal to its algebraic multiplicity.
\begin{solution}

\end{solution}

\question[1] 
Prove if A and B are positive definite then so is A + B. 
\begin{solution}

\end{solution}

\uplevel{\textbf{Eigenstory: Special Matrices}}
\question[2] 
Consider the matrix $R = I - 2\mathbf{u}\mathbf{u}^\top$ where $\mathbf{u}$ is a unit vector $\in \mathbb{R}^n$.

\begin{parts}
\part Show that $R$ is symmetric and orthogonal. (How many independent vectors will $R$ have?)
\begin{solution}

\end{solution}


\part Let $\mathbf{u} = \frac{1}{\sqrt{2}}\begin{bmatrix} 1 \\ 0\\ 1 \end{bmatrix}$. Draw the line passing through this vector in geogebra (or any tool of your choice). Now take any vector in $\mathbf{R}^3$ and multiply it with the matrix $R$ (i.e., the matrix $R$ as defined above with $\mathbf{u} = \frac{1}{\sqrt{2}}\begin{bmatrix} 1 \\ 0\\ 1 \end{bmatrix}$). What do you observe or what do you think the matrix $R$ does or what would you call matrix $R$? (Hint: the name starts with $R$)
\begin{solution}

\end{solution}

\part Compute the eigenvalues and eigenvectors of the matrix $R$ as defined above with $\mathbf{u} = \frac{1}{\sqrt{2}}\begin{bmatrix} 1 \\ 0\\ 1 \end{bmatrix}$
\begin{solution}

\end{solution}

\part I believe that irrespective of what $\mathbf{u}$ is any such matrix $R$ will have the same eigenvalues as you obtained above (with one of the eigenvalues repeating). Can you reason why this is the case? (Hint: think about how we reasoned about the eigenvectors of the projection matrix $P$ even without computing them.)
\begin{solution}

\end{solution}

\end{parts}

\question[2] 
Let $Q$ be a $n \times n$ real orthogonal matrix (i.e., all its elements are real and its columns are orthonormal). State with reason whether the following statements are True or False (provide a proof if the statement is True and a counter-example if it is False).

\begin{parts}
\part If $\lambda$ is an eigenvalue of $Q$ then $\lambda^2 = 1$. (0.5 marks)
\begin{solution}

\end{solution}
\part The eigen vectors of $Q$ are orthogonal. Just state yes or no.(0.25 marks)
\begin{solution}

\end{solution}
\part $Q$ is always diagonalizable, and if it is diagonisable only under some particular condition, give prove for that.(1.25 marks)
\begin{solution}

\end{solution}
\end{parts}


\question[1\half] 
Any rank one matrix can be written as $\mathbf{u}\mathbf{v}^\top$. 
\begin{parts}
\part Prove that the eigenvalues of any rank one matrix are $\mathbf{v}^\top\mathbf{u}$ and $0$.
\begin{solution}

\end{solution}

\part How many times does the value $0$ repeat? 
\begin{solution}

\end{solution}

\part What are the eigenvectors corresponding to these eigenvalues?
\begin{solution}

\end{solution}
\end{parts}


\question[2] 
Consider a $n \times n$ Markov matrix.
\begin{parts}
\part Prove that the dominant eigenvalue of a Markov matrix is 1
\begin{solution}
\\
Proof (part 1): 1 is an eigenvalue of a Markov matrix \\
Proof (part 2): all other eigenvalues are less than 1\\
(If you have a simpler way of proving this instead of proving it in two parts then feel free to do so but your proof should convince me about both these parts.)
\end{solution}

\part Consider any $2\times2$ matrix $\begin{bmatrix}a & b \\ c& d \end{bmatrix}$ such that $a + b = c + d$. Show that one of the eigenvalues of such a matrix is 1. (I hope you notice that a Markov matrix is a special case of such a matrix where  $a + b = c + d = 1$.)
\begin{solution}
\end{solution}

\part Does the result extend to $n \times n$ matrices where the sum of the elements of a row is the same for all the $n$ rows? (Explain with reason)
\begin{solution}

\end{solution}

\part What is the corresponding eigenvector?
\begin{solution}

\end{solution}

\end{parts}


\uplevel{\textbf{Eigenstory: Special Relations}}

\question[4] 
For each of the statements below state True or False with reason.

\begin{parts}

\part If i(complex number) is an eigen value of A , then it follows that i is  an eigen value of $A^{-1}$.
\begin{solution}

\end{solution}
\part If the characteristic equation of a matrix A is 
$\lambda^5+7\lambda^3-6\lambda^2+128=0$ then sum of eigen values is -7.
\begin{solution}

\end{solution}
\part If A is $3 \times 3$ matrix with eigenvector as $\begin{bmatrix} 4 \\ -3\\ 2 \end{bmatrix}$ then $\begin{bmatrix} 16 \\ -12\\ 8 \end{bmatrix}$ is also an eigen vector of A.
\begin{solution}

\end{solution}

\part If A is symmetric matrix then the algebraic and geometric multiplicity is same for every eigen value.
\begin{solution}

\end{solution}

\part If $\mathbf{x}$ is an eigenvector of $A$ and $B$ then it is also an eigenvector of both $AB$ and $BA$, even if the eigenvalues of $A$ and $B$ corresponding to $\mathbf{x}$ are different.
\begin{solution}

\end{solution}
\part If $\mathbf{x}$ is and eigenvector of $A$ and $B$ then it is also an eigenvector of $A+B$
\begin{solution}

\end{solution}

\part The non-zero eigenvalues of $AA^\top$ and $A^\top A$ are equal.
\begin{solution}
~\\
\end{solution}
\part The eigenvectors of $AA^\top$ and $A^\top A$ are always same.
\begin{solution}
~\\
\end{solution}
\end{parts}

\uplevel{\textbf{Eigenstory: Change of basis}} 

\question[2] 
Consider the following two basis. Basis 1: $\mathbf{u_1} = \frac{1}{5}\begin{bmatrix} 3 \\ 4 \end{bmatrix}, \mathbf{u_2} = \frac{1}{5}\begin{bmatrix} 4 \\ -3 \end{bmatrix},$ and Basis 2: $\mathbf{u_1} = \frac{1}{5}\begin{bmatrix} 3 \\ -4 \end{bmatrix}, \mathbf{u_2} = \frac{1}{5}\begin{bmatrix} -4 \\ -3 \end{bmatrix},$. Consider a vector $\mathbf{x} = \begin{bmatrix} a \\ b \end{bmatrix}$ in Basis 1 (i.e., $\mathbf{x} = a \mathbf{u_1} + b \mathbf{u_2}$). How would you represent it in Basis 2?
\begin{solution}

\end{solution}

\question[1] 
Let $\mathbf{u}$ and $\mathbf{v}$ be two vectors in the standard basis. Let $T(\mathbf{u})$ and $T(\mathbf{v})$ be the representation of these vectors in a different basis. Prove that $\mathbf{u}\cdot\mathbf{v} = T(\mathbf{u})\cdot T(\mathbf{v})$ if and only if the basis represented by $T$ is an orthonormal basis (i.e., dot products are preserved only when the new basis is orthonormal).
\begin{solution}

\end{solution}

\uplevel{\textbf{Eigenstory: PCA and SVD}}
\question[1] 
We are familiar with the following equation: $A=U \sum V^T$, where A is a real valued $m\times n$ matrix and other symbols have their usual meanings. State how this equation is related to Principal Component Analysis. State the correct dimensions of the three matrices at the RHS of the given equation. (No vague answers please.)
\begin{solution}

\end{solution}

\question[1\half] 
Consider the matrix $\begin{bmatrix} 5 & 5 \\ -4 & 4
\end{bmatrix}$

\begin{parts}
\part Find $\Sigma$ and $V$, \textit{i.e.}, the eigenvalues and eigenvectors of $A^\top A$
\begin{solution}

\end{solution}
\part Find $\Sigma$ and $U$, \textit{i.e.}, the eigenvalues and eigenvectors of $A A^\top$
\begin{solution}

\end{solution}
\part Now compute $U\Sigma V^\top$. Did you get back $A$? If yes, good! If not, what went wrong?
\begin{solution}
Please refer to following lectures of Prof. Gilbert Strang to understand what went wrong and then correct your answer (if it was wrong): 
\begin{itemize}
    \item \url{https://www.youtube.com/watch?v=TX_vooSnhm8&t=1177s} (starts at 1177 seconds)
    \item \url{https://www.youtube.com/watch?v=HgC1l_6ySkc&feature=youtu.be&t=1731)} (starts at 1731 seconds)
\end{itemize}
\end{solution}

\end{parts}

\question[2] 
Prove that the matrices $U$ and $V$ that you get from the SVD of a matrix $A$ contain the basis vectors for the four fundamental subspaces of $A$. (this is where the whole course comes together: fundamental subspaces, basis vectors, orthonormal vectors,  eigenvectors, and our special symmetric matrices $AA^\top$, $A^\top A$!)
\begin{solution}

\end{solution}


~\\...And that concludes the story of \textit{How I Met Your Eigenvectors :-)} (Hope you enjoyed it!)


\vspace{5mm}

\par\noindent\rule{\textwidth}{0.4pt}

\vspace{3mm}

\textbf{EXTRA QUESTIONS FOR PRACTICE}. These questions will not be evaluated for grades, but we encourage you to solve them to gain better understanding of the concepts. 


\question[0] Prove that for any square matrix $A$ the eigenvectors corresponding to distinct eigenvalues are always independent. 


\question[0] Prove the following. 
\begin{parts}
\part The sum of the eigenvalues of a matrix is equal to its trace.


\part The product of the eigenvalues of a matrix is equal to its determinant.


\end{parts}

\question[0] What is the relationship between the rank of a matrix and the number of non-zero eigenvalues? Explain your answer.
\begin{solution}
I think the answer to this question is ``The rank of a matrix is equal to the number of non-zero eigenvalues if $\cdots$''
\end{solution}


\question[0] If $A$ is a square symmetric matrix then prove that the number of positive pivots it has is the same as the number of positive eigenvalues it has. 



\question[0] For each of the statements below state True or False with reason.

\begin{parts}

\part If $\mathbf{x}$ is an eigenvector of $A$ and $B$ then it is also an eigenvector of both $AB$ and $BA$, even if the eigenvalues of $A$ and $B$ corresponding to $\mathbf{x}$ are different.


\part If $\mathbf{x}$ is and eigenvector of $A$ and $B$ then it is also an eigenvector of $A+B$




\part The non-zero eigenvalues of $AA^\top$ and $A^\top A$ are equal.


\end{parts}

\question[0] How are PCA and SVD related? (no vague answers please, think and answer very precisely with mathematical reasoning)





\question[0] Fun with Objects.
\begin{parts}
\part In this activity, you need to find four different rank one objects and paste their photos. For e.g. the flag of Russia is a rank one flag. You can use an object of the same type only once, for e.g. you cannot use flags twice. Also avoid matrices and flag of Russia as answer.


\part What is the rank of a hypothetical 4 x 8 chess board?


\end{parts}

\question[0] Consider the LFW dataset (Labeled Faces in the Wild). 

\begin{parts}
\part Perform PCA using this dataset and plot the first 25 eigenfaces (in a $5 \times 5$ grid) 

\begin{solution}
Here is something to get you started.

\begin{verbatim}
import matplotlib.pyplot as plt
from sklearn.datasets import fetch_lfw_people
from sklearn.decomposition import PCA


# Load data
lfw_dataset = fetch_lfw_people(min_faces_per_person=100)

_, h, w = lfw_dataset.images.shape
X = lfw_dataset.data

# Compute a PCA 
n_components = 100
pca = PCA(n_components=n_components, whiten=True).fit(X)

Beyond this you are on your own. Good Luck!
\end{verbatim}
\end{solution}

\part Take your close-up photograph (face only) and reconstruct it using the first 25 eigenfaces :-). If due to privacy concerns, you do not want to to use your own photo then feel free to use a publicly available close-up photo (face only) of your favorite celebrity.


\end{parts}

\end{questions}
\end{document} 